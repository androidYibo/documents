% \documentclass{article}

% % \usepackage[utf8]{inputenc} %Stationær ÆØÅ
% \usepackage[ansinew]{inputenc} %Bærbar ÆØÅ?

% %\usepackage{url} % Allows hyperlinks
% \usepackage{minted}
% \usepackage[hyphens]{url} %URLs
% \usepackage{graphicx} % Allows figures
% \usepackage{etoolbox} %for configuration of sloppy
% \usepackage{tabularx}
% %Section style
% \usepackage{xcolor}


% \definecolor{secnum}{RGB}{102,102,102}

% \makeatletter
    % \def\@seccntformat#1{\llap{\color{secnum}\csname the#1\endcsname\hskip 16pt}}
% \makeatother
% %end section style

% {\sloppy}{\hbadness 10000\relax}{}{} %adds hbadness to sloppy
% \setlength{\paperheight}{297mm} %Sets the page to an A4
% \setlength{\paperwidth}{210mm}        %Sets the page to an A4

\begin{document}

\begin{titlepage}
\begin{center}
\textsc{\Large Haskell Quant Library}\\[0.5cm]
\textsc{Synopsis}\\[0.5cm]
\vspace{2 cm}
\begin{tabular}{ll}
Student: & Kasper Passov\\
Supervisors: & Fritz Henglein \\ 
             & Jost Berthold
\end{tabular}
\end{center}
\vspace{5 cm}
\newpage
\tableofcontents
\end{titlepage}

\section{Project Title}
Haskell Library for Quantitative Analysis 

\section{Problem Definition}

I want to design and develop a Haskell library for 
quantitative finance, taking the open source quantlib
as a starting point. The language Haskell is used
because of its purity and advanced type class system
which allows it to model relevant entities.\\
The result of my project should be a software
architecture that supports different kinds of 
financial instruments and valuation methods.

\section{Project Constraints}

TODO: Find the freaking constraints

\section{Project Justification}

TODO: further describition needed\\
\\
\subsection{Reasoning for module (TAKEN FROM Quantlib^\cite{QULI})}
Such a library could greatly improve the workout of quants (do i need to include this?).\\
Few good resources for quantative libraries appart from Quadlib\\
\\
\subsection{Reasoning for Haskell (TAKEN FROM HQL^\cite{HQL})}
modularity\\
laziness\\
Haskell is a pure language like the math behind the methods making implementation easier.\\
Functional programming, making it easier to reuse functions for multiple tasks\\
Statically type, errors on compile and overload types (not 100\% sure how this works yet, is this the type families?)\\


\section{Schedule}

\subsection{Timeline}

TODO: more precision based on constraints, also description of legends

\begin{ganttchart}{1}{17}
    \gantttitle{week}{17} \\
    \gantttitlelist{7,...,23}{1} \\
    \ganttgroup{Pre implementation}{1}{3} \\
    \ganttgroup{Implementation}{4}{10} \\
    \ganttgroup{Report}{10}{15} \\
    \ganttgroup{Fine Tuning}{16}{17} \\
    \ganttnewline[thick]
    \ganttbar[bar height=.7]{Synopsis}{1}{2} \\
    \ganttbar[bar height=.7]{Reading}{1}{3}\\
    \ganttmilestone{Ready for Implementation}{3}\\
    \ganttbar[bar height=.7]{Implementation}{4}{9} \\
    \ganttbar[bar height=.7]{Testing and documentation}{8}{10} \\
    \ganttmilestone{Implementation done}{10}\\
    \ganttbar[bar height=.7]{Write Report}{10}{15} \\
    \ganttmilestone{Report done}{15}\\
    \ganttbar[bar height=.7]{Fine tuning}{15}{17} \\
    \ganttmilestone{Everything done}{17}\\
    \ganttlink{elem5}{elem6}
    \ganttlink{elem6}{elem7}
    \ganttlink{elem7}{elem8}
    \ganttlink{elem8}{elem9}
    \ganttlink{elem9}{elem10}
    \ganttlink{elem10}{elem11}
    \ganttlink{elem11}{elem12}
    \ganttlink{elem12}{elem13}
\end{ganttchart}
                                                                            
\subsection{Work Activities and tasks}
Activities and tasks as described in OOSE\cite{OOSE}\\
TODO: more serious description of activities
and tasks.  

\subsubsection{Activities}

\paragraph{Pre implementaion}
Before i implement the code i need a lot of knowledge.

\paragraph{Implementaion}
The time used to implement the project describer in constraints

\paragraph{Report}
The time to write everything i think i know about my code. (maybe let this
overlap more with the implementation?)

\paragraph{Fine Tuning}
Wastetime so the schedule has room for sliding

\subsubsection{Tasks}

\paragraph{Synopsis}
Write a synopsis
\paragraph{Reading}
Read stuff when needed. Proberly a lot longer
\paragraph{Implementation}
Implement the code. Needs to be split in smaller
pieces when i have some constraints
\paragraph{Testing and documentation}
Write and run tests on the implementation, and
write documentation
\paragraph{Write report}
Write the report, also needs to be split further
\paragraph{Fine tuning}
Time to correct any and all spelling errors, and maybe 
any coding and logical errors if i did not get that right
the first time

\begin{thebibliography}{9}
\bibitem{OOSE}
    Bernd Bruegge, Allen H.Dutoit
    \emph{Object-Oriented Software Engineering Using UML, Patterns and Java},
    Person New International Edition,
    Third Edition,
    2014.
\bibitem{QULI}
    Luigi Ballabio
    \emph{Implementing Quantlib},
    Draft,
    2013.
\bibitem{HQL}
    Andreas Bock, johan Astborg, Jost Berthold, Sinan Gabel
    \emph{HIPERFIT Quant Library},
    2014.
\end{thebibliography}
\end{document}

