\documentclass[10pt,a4paper]{article}
\usepackage{ap}

\title{Exam solution for Advanced Programming 2013}
\author{Niels G. W. Serup}
\setupdocument
\begin{document}
\adjustdocument
\maketitle
\tableofcontents


\newpage
\section{The Salsa Language}

All Salsa-related source files are in \verb+src/salsa+.


\subsection{Grammar modification}

Operator associativity is left unspecified in the grammar.  I have chosen to
assume left associativity where the choice between left and right affect the
semantics, i.e. in \texttt{Expr '-' Prim}, and accept the ambiguity where the
semantics are unaffected, i.e. in \texttt{Expr '+' Prim} and \texttt{Command 
'||' Command}.

I keep it ambigious because I think only the semantics of parsed programs are
important.

Specifically, with \texttt{Expr '-' Prim} you can have \texttt{2 - (1 - 1) != (2
  - 1) - 1}, but the same is not the case for \texttt{+} and \texttt{||}.


\subsection{Parser}

My parser uses Parsec.  It assumes left associativity where I have left the
associativity ambiguous.

Most of the parser is fairly straightforward.  The only interesting pieces of
code are those handling precedence and associativity in \verb+command'+ and
\verb+expr'+.  I have used Parsec's \verb+chainl1+ to handle \texttt{Command
  '||' Command}, and my own \verb+chainpre1+ to handle \texttt{Command '@'
  Vident}, \texttt{Expr '+' Prim}, and \texttt{Expr '-' Prim}.

\verb+chainl1+ is used to run the parsers of highest precedence first and then
join them with lower precedence operators.  It runs on expressions of the type
\verb+a ::= a op a+. \verb+chainpre1+ does the same, except it works on
expressions of the type \verb+a ::= a op b+ instead.


\subsection{Tests}

To test the correctness of the parser, it is necessary to keep in mind the
ambigious grammar.  What's important is that the semantics of a parsed program
are right, so instead of testing by comparing parsed programs to ASTs, I have
tested by comparing interpreted parsed program to animations.  The downside is
that the tests depend on the correctness of the interpreter, but there is
no other practical way unless the grammar is made unambigious.

I have made two sets af manual tests and one QuickCheck property in
\verb+SalsaParserTests.hs+.

The first set of manual tests checks the animation interpreted from a parsed
string.  The second set tests that strings with wrong grammar are not
succesfully parsed.

The QuickCheck property generates a program AST, formats it to a string, parses
the string into a new AST, interprets both ASTs and compares the resulting
animations.  This is done 1000 times.  Read more about my program generator in
section 4.

All tests pass, so my implementation is probably sound.  However, I cannot be
sure until I have tested my interpreter, since it is used in the parser tests.



\newpage
\section{Interpreting Salsa}

In my implementation, a shape is only drawn if a move happens, i.e. an animation
with just one key frame has 0 frames.  I don't think this was clear from the
exam text, but my way seems the simplest and fits nicely with the required
functionality of \verb+interpolate+.


\subsection{Salsa and SalsaCommand monads}

These are my main types:
\begin{minted}[frame=lines]{haskell}
type Shape = Position -> Ident -> GpxInstr

data Env = Env { views :: [(Ident, Integer, Integer)]
               , groups :: [(Ident, [Ident])]
               , shapes :: M.Map Ident (M.Map Ident Shape)
               , activeViews :: [Ident]
               , envFps :: Integer
               }

-- | The positions of the shapes in the views.
type ShapePosMap = M.Map Ident (M.Map Ident Position)

data Context = Context { contextEnv :: Env
                       , contextSPM :: ShapePosMap
                       }

newtype SalsaCommand a = SalsaCommand {
  runSC :: Context -> (ShapePosMap, [Frame], a) }

newtype Salsa a = Salsa { runS :: Context -> (Context, [Frame], a) }
\end{minted}

\verb+Env+, \verb+ShapePosMap+ and \verb+Context+ holds what's required of them.
\verb+SalsaCommand+ and \verb+Salsa+ are almost the same (also in their monad
instances), the only difference being that \verb+Salsa+ can modify all of the
\verb+Context+, and that \verb+SalsaCommand+ can only modify the
\verb+ShapePosMap+.  This is because a Salsa \verb+Command+ should only move
shapes and generate frames, not change the context in any way.

Since \verb+Salsa+ and \verb+SalsaCommand+ are so alike, they can share several
useful functions.  For this reason I have made a typeclass \verb+SalsaSomething+
with the functions \verb+getContext+ and \verb+modifySPM+ which is instanced
for both types.


\subsection{Generating frames}

The interesting part of the interpreter is how it generates frames.
\verb+runProg+ starts a \verb+Salsa+ monad which, as long as there are
\verb+Definition+s, just modifies the context state again and again.  When it
sees a \verb+Command+, the \verb+SalsaCommand+ monad generates frames which are
then appended to the existing frames in the \verb+Salsa+ monad state.

For a given \verb+Command+, \verb+SalsaCommand+ generates frames this way:

\begin{enumerate}
\item Save the old shape positions.
\item Create as many parallel commands as needed so that every command is
  simplified to a move of just one shape.  If there is an \verb+At+, it needs to
  be run in another environment, which is done by the \verb+withEnv+ function.
\item For each shape move, calculate and set the new position for the shape in
  all its active views.
\item For each view, for each shape in that view, interpolate the frames to the
  next key frame, and add them to the \verb+SalsaCommand+ state.
\item Merge all the frame lists so that all lists will be executed concurrently
  when run in an animation.
\end{enumerate}

The bind operation of the monad concatenates frames, so to merge them instead I
have created another function, \verb+joinFrames+, which is similar to \verb+>>+
except for merging.


\subsection{Tests}

I have created a set of manual tests and a set of QuickCheck properties in
\verb+SalsaInterpTests.hs+.

The manual tests each take an fps number, a program string, and the expected
\verb+Animation+ and checks that the parsed program interpreted at the given fps
results in the expected animation.  I have made manual tests for interpolating,
shapes, multiple moves, views and groups, the \verb+@+ and \verb+||+ operators,
and \verb+Expr+.

The QuickCheck properties test the invariants:
\begin{itemize}
\item \verb+interpolate n p0 p1+ must return a sorted list where the last
  element is \verb+p1+ if \verb+n /= 0+, and the length is \verb+n+.
\item The number of frames must be equal to the number of non-parallel moves
  times the framerate.
\item All views of an AST must also be in the animation.
\item When a shape occurs in the AST, it must also occur in at least one frame
  in the animation (because the program generator makes sure to move all shapes
  at least once).
\item When a move-to-position is absolute, there must be at least one frame in
  which a shape is at that position.
\end{itemize}

All tests pass.  However, some of the tests depend on the correctness of the
parser, and since some of the parser tests depend on the correctness of the
interpreter, there is a co-dependency problem here.  But since the QuickCheck
properties in both the parser and interpreter are checked alongside manual tests
which show that their outputs are clearly different for different inputs, it's
probably not a problem, so my implementation is probably sound.

All Salsa parser and interpreter tests can be run with \verb+test.hs+.  You can
also run \verb+animate.hs+ to generate animations from files.


\newpage
\section{Atomic Transaction Server}

The source files are in \verb+src/at_server+.

I use OTP in my implementation.  My transaction server is split up into two
FSMs: the master (\verb+at_server_master.erl+) and the helper
(\verb+at_server_helper.erl+).  In \verb+at_server.erl+, the master is directly
used for \verb+start/1+, \verb+stop/1+, \verb+doquery/2+, and \verb+begin_t/1+,
while helpers are used for \verb+query_t/3+, \verb+update_t/3+ and
\verb+commit_t/2+.


\subsection{Master and helper functionality}

There is only one state name, \verb+ready+.  When \verb+at_server:start(State)+
is called, \verb+State+ is stored in the master FSM, and can only changed with a
commit message from a helper.

Even though the master aborts all transactions when accepting a commit from a
helper process, another helper process might still have had time to send the
master another commit.  To ensure that that commit is refused, the master and
the helpers maintain \verb+Iter+ states.

When a transaction is created, the master's \verb+Iter+ is copied to the helper,
and when the helper tries to commit, the master only accepts it if their
\verb+Iter+s are equal.  Once accepted, the master increases its \verb+Iter+
with 1, thus refusing any further commits from currently dying processes.

The helper also has only state name, \verb+ready+.  Like the master it maintains
\verb+State+ and \verb+Iter+ states, but it also keeps a reference to its
master.


\subsection{PIDs instead of refs}

In my implementation, transactions are referred to by PIDs instead of refs.  It
seems easier and smarter than maintaining a dictionary in the master and having
to query the master every time a helper is to run a command.

One upside of the dictionary approach is that a transaction not created by the
current master cannot commit to the current master.  With direct command
passing, that upside isn't there.  I have corrected this by making checks in all
helper functions used by the API that check whether the internal master pid
equals the master pid in the API function call (like
\verb+at_server:update_t(Master, ...)+

On reflection, I realize that while refs created by \verb+make_ref+ are
unique\footnote{\url{http://www.erlang.org/doc/man/erlang.html}}, I don't know
about process ids.  If a new process takes over the id of a previous process,
that might cause problems, but it doesn't appear to be a problem in my tests.


\subsection{Limitations in \texttt{at\_extapi}}

The requirement that functions in \verb+at_extapi.erl+ must contact the server
via \verb+at_server.erl+ means that \verb+abort+ and \verb+ensureUpdate+ are not
as efficient as they could be.

To abort a transaction, \verb+abort+ has to exploit the fact that a transaction
must fail if \verb+at_server:update_t/3+ fails with an error.  To do this, it
sends a function to \verb+update_t+ which always fail.  A simpler way would be
to just kill the transaction with \verb+exit(Helper, kill)+.

The way \verb+ensureUpdate+ works is that it calls \verb+tryUpdate+ again and
again until it returns \verb+ok+ or \verb+error+.  A simpler, less forceful way
would be to extend the main API to accept updates directly on the master,
although that would remove the concurrency.

It is not specified in the exam text what to return in \verb+choiceUpdate+ if
all return values are \verb+error+.  In my implementation, the final return
value is then also \verb+error+.


\subsection{Ensuring \texttt{aborted}}

Assuming that no new process takes the pid of a previous process, a call to an
aborted transaction will always return \verb+aborted+, since the implementation
catches \verb+noproc+ errors.  However, a transaction might not \textbf{be}
aborted at once, since \verb+at_extapi:abort/2+ is used to abort the processes.


\subsection{Tests}

I have added type specifications to most functions with \verb+-spec+, although I
haven't tested them with Dialyzer; I put them there mostly for my own sake.

The file \verb+at_server_tests.erl+ contains 6 hand-crafted tests of the
transaction server which all pass.  The tests use the EUnit testing
framework\footnote{\url{http://www.erlang.org/doc/apps/eunit/chapter.html}} and
can be run with the generated function \verb+at_server_tests:test/0+.

They test basic function application, sequential transactions, concurrent
transactions, \verb+tryUpdate+, \verb+ensureUpdate+, and \verb+choiceUpdate+.
That covers all of the API, so it seems my implementation is correct.



\newpage
\section{Salsa program generator}

See \verb+SalsaGenerator.hs+.

For some of the Salsa tests to work, there must be a program generator to create
random Salsa ASTs.  For some tests there must also be a program text generator
to transform an AST into a string.

To create the random ASTs, I use QuickCheck's \verb+Test.QuickCheck.Gen+ module.
The generation is fairly straightforward, except that the generated programs
need to be correct for some of the tests to make sense.  To ensure correctness,
the generator keeps a environment like in the interpreter (but simpler), so that
it always knows which shapes it can move, etc.

Creating a program string from an AST is straightforward, though it's important
to remember that the left associativity in \verb+Expr '-' Prim+ might make it
necessary to put parantheses around an expression.  There are infinitely many
correct program texts matching a single AST because expressions can be in
arbitrarily many parentheses, and because there can be arbitrarily many
whitespace characters.  My random program string generator uses this to create
strings with few or many spaces, parentheses and brackets, so that the parser
can be fed differently formatted versions of the same program.


\newpage
\appendix
\section{Source Code}

\subsection{Salsa}

\subsubsection{General}

\inputminted[frame=lines, label=Gpx.hs]{haskell}{../src/salsa/Gpx.hs}

\inputminted[frame=lines, label=Misc.hs]{haskell}{../src/salsa/Misc.hs}

\inputminted[frame=lines, label=SalsaAst.hs]{haskell}{../src/salsa/SalsaAst.hs}

\inputminted[frame=lines, label=Salsa.hs]{haskell}{../src/salsa/Salsa.hs}


\subsubsection{Parser}

\inputminted[frame=lines, label=SalsaParser.hs]{haskell}{../src/salsa/SalsaParser.hs}


\subsubsection{Interpreter}

\inputminted[frame=lines, label=SalsaInterp.hs]{haskell}{../src/salsa/SalsaInterp.hs}


\subsubsection{Tests}

\inputminted[frame=lines, label=SalsaGenerator.hs]{haskell}{../src/salsa/SalsaGenerator.hs}

\inputminted[frame=lines, label=SalsaParserTests.hs]{haskell}{../src/salsa/SalsaParserTests.hs}

\inputminted[frame=lines, label=SalsaInterpTests.hs]{haskell}{../src/salsa/SalsaInterpTests.hs}


\subsubsection{Scripts}

\inputminted[frame=lines, label=animate.hs]{haskell}{../src/salsa/animate.hs}

\inputminted[frame=lines, label=test.hs]{haskell}{../src/salsa/test.hs}



\subsection{AT Server}

\subsubsection{API}

\inputminted[frame=lines, label=at\_server.erl]{erlang}{../src/at_server/at_server.erl}

\inputminted[frame=lines, label=at\_extapi.erl]{erlang}{../src/at_server/at_extapi.erl}


\subsubsection{Implementation}

\inputminted[frame=lines, label=at\_server\_master.erl]{erlang}{../src/at_server/at_server_master.erl}

\inputminted[frame=lines, label=at\_server\_helper.erl]{erlang}{../src/at_server/at_server_helper.erl}


\subsubsection{Tests}

\inputminted[frame=lines, label=at\_server\_tests.erl]{erlang}{../src/at_server/at_server_tests.erl}


\end{document}
